\begin{huge}
\textbf{Code Pieces}
\end{huge}



\section{\uppercase\expandafter{\romannumeral1}. Atomic swaps based on HTLC}

\begin{lstinputlisting}[language=JavaScript,  
						caption=Bitshares side, 	
						firstline = 10, 
						lastline = 50,
						escapechar=@]{./codes/bts.js}
\end{lstinputlisting}
see line~\ref{line:h1}

\begin{lstinputlisting}[language=JavaScript, 
						caption=Ethereum Side, 
						firstline=20,
						lastline=50]{./codes/eth.js}	
\end{lstinputlisting}

\section{\uppercase\expandafter{\romannumeral2}. Ethereum smart contracts}
\label{sec:contract}


\begin{lstlisting}[language=Solidity, 
						caption = Create hash timelock contract example]
pragma solidity ^0.5.0;

contract HashTimelock{
    event LogHTLCNew(
        bytes32 indexed contractId,
        address indexed sender,
        address indexed receiver,
        uint amount,
        bytes32 hashlock,
        uint timelock
    );
	...
	
	modifier fundsSent() {
        require(msg.value > 0, "msg.value must be > 0");
        _;
    }
    modifier futureTimelock(uint _time) {
        require(_time > now, "timelock time must be in the future");
        _;
    }
    ...
    
    function newContract(address payable _receiver, bytes32 _hashlock, uint _timelock)
        external
        payable
        fundsSent
        futureTimelock(_timelock)
        returns (bytes32 contractId)
    {
        contractId = sha256(
            abi.encodePacked(
                msg.sender,
                _receiver,
                msg.value,
                _hashlock,
                _timelock
            )
        );

\end{lstlisting}
\noindent This contract provides a way to create and keep
HTLCs for ETH.\footnotemark[1] Detail protocol:

\begin{enumerate}
	\item \texttt{newContract(receiver, hashlock, timelock)}- sender calls this to create a new HTLC and gets back a 32 byte contract id
	\item \texttt{withdraw(contractId, preimage)} - once the receiver knows the preimage of the hashlock hash they can claim the ETH with this function
	\item \texttt{refund()} - after timelock has expired and if the receiver did not withdraw funds the sender / creator of the HTLC can get their ETH back with this function.
\end{enumerate}
\footnotetext[1]{The complete codes can be found at \url{https://github.com/Fy45/BTS-ETH-atomic_swaps/blob/master/contract/HashedTimelock.sol}}

\begin{lstinputlisting}[language=Solidity, 
						firstline=20, 
						lastline=50, 
						caption= Ethereum micro-payment channel -- Machinomy]{./codes/contract.sol}
	
\end{lstinputlisting}



