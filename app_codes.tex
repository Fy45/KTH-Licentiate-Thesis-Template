\begin{huge}
\textbf{Code Pieces}
\end{huge}


\input{format/solidity-highlighting.tex}
\input{format/highlight_js.tex}
\input{format/jsonHighlight}


\section{\uppercase\expandafter{\romannumeral1}. Atomic swaps based on HTLC}

\begin{lstinputlisting}[language=JavaScript,  
						caption=Bitshares side, 	
						firstline = 10, 
						lastline = 50,
						escapechar=@]{./codes/bts.js}
\end{lstinputlisting}
see line~\ref{line:h1}

\begin{lstinputlisting}[language=JavaScript, 
						caption=Ethereum Side, 
						firstline=20,
						lastline=50]{./codes/eth.js}	
\end{lstinputlisting}

\section{\uppercase\expandafter{\romannumeral2}. Ethereum smart contracts}

\noindent This contract provides a way to create and keep
HTLCs for ETH. Detail protocol:

\begin{enumerate}
	\item \texttt{newContract(receiver, hashlock, timelock)}- sender calls this to create a new HTLC and gets back a 32 byte contract id
	\item \texttt{withdraw(contractId, preimage)} - once the receiver knows the preimage of the hashlock hash they can claim the ETH with this function
	\item \texttt{refund()} - after timelock has expired and if the receiver did not withdraw funds the sender / creator of the HTLC can get their ETH back with this function.
\end{enumerate}
  

\begin{lstinputlisting}[language=Solidity, 
						firstline = 12, 
						lastline =40, 
						caption = Hashed Timelock Contracts (HTLCs) on Ethereum]{./codes/HashedTimelock.sol}

\end{lstinputlisting}


\begin{lstinputlisting}[language=Solidity, 
						firstline=20, 
						lastline=50, 
						caption= Ethereum micro-payment channel -- Machinomy]{./codes/contract.sol}
	
\end{lstinputlisting}



