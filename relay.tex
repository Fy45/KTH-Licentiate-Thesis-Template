\subsection{Relay Scheme}
\subsubsection{BTC-Relay}
\noindent In 2016, the BTC-Relay released by the Consensys team is the most classic relay cross-chain solution, enabling cross-chain transactions between Ethereum and Bitcoin, as well as realizing Ethereum's DApp applications to support BTC payments. Since Bitcoin scripts are non-turing complete and difficult to support complex applications, BTC-Relay only implements one-way cross-chain from Bitcoin to Ethereum.
        \begin{figure}[H]
        \includegraphics[width=1\textwidth]{./figures/btc.png}
        \centering
        \caption{BTC-Relay cross-chain process diagram}%\protect\footnotemark}
        \centering
        \label{fig:btc}
        \end{figure}
\noindent Like Figure \ref{fig:btc} shows, BTC-relay itself is a smart contract for Ethereum. The function of the contract is to verify certain transactions on Bitcoin and provide verification information to other DApp users on the Ethereum. Relay is a group of users who obtain block header data from Bitcoin and has the account address of the Ethereum network. The Relay that submits the block header data to the BTC-Relay contract as soon as possible can get the ETF transaction fee reward. After obtaining the block header data, the BTC-Relay smart contract can verify a transaction according to the principle of SPV proof. When a transaction in the Bitcoin network does occur, it can trigger the specific transaction or smart contract execution of the Ethereum network.

\subsubsection{Cosmos}
\noindent Cosmos\cite{cosmos} is a cross-chain platform project initiated by the Tendermint team in 2017. It supports the modular establishment of Cosmos isomorphism chain and also supports the external heterogeneous chain through Bridge. Its most important feature is that all the chains in the Cosmos system are isomorphic chains and can more easily support the flow of assets across the chain. All the zones share a set of network protocols, consensus mechanisms, and data storage methods. Assemble the new Zone blockchain through the API interface.\\
\noindent The overall architecture of Cosmos network are shown in Figure\ref{fig:cosmos}:
        \begin{figure}[H]
        \includegraphics[width=0.7\textwidth]{./figures/cosmos.png}
        \centering
        \caption{Cosmos network architecture}%\protect\footnotemark}
        \centering
        \label{fig:cosmos}
        \end{figure}
        
\noindent There are many Zones connected to the Hub (Hub is a chain, and each Zone is also a chain). Cosmos Hub maintains a multi-asset distributed ledger and masters the asset status of all the Zones that connected to it. Each zone will synchronize the status of the Hub, but the communication between the Zone and the Zone can only be done indirectly through the Hub. Each cross-chain asset transfer requires a common confirmation of the zone, hub, and receiving zone to be successful. 
\par The information is transmitted between zones through packets based on the IBC (Inter-Blockchain Communication) protocol. The blocks in a space pack the data to be delivered into standard IBC data packets and finally complete the transmission through the network layer UDP or TCP protocol.
        \begin{figure}[H]
        \includegraphics[width=0.8\textwidth]{./figures/IBC.png}
        \centering
        \caption{IBC sequence diagram}%\protect\footnotemark}
        \centering
        \label{fig:IBC}
        \end{figure}
\begin{enumerate}
    \item  Zone 1 initiates an IBCBlockCommitTx transaction, passing the new block header information (including all Validator's public key) to the HUB;
    \item  Zone 1 initiates an assets transaction;
    \item Zone 1 verifies the transaction;
    \item Send the valid transaction into the message queue for HUB;
    \item Zone 1 listens to a new message in the queue, which generates Merkle proof, and sent to HUB as IBCPacketTx's Payload. (In each space there is an independent third-party relay program that generates Merkle Proof from the original chain and assembles it into a Packet and initiates the transaction, passing it to the receiving chain);
    \item HUB verifies that Merkle Proof is valid, if it is valid, send a message to Zone2 (The process of sending a message to Zone2 by HUB is the same as steps 1~6);
    \item Zone2 verifies that Zone1 is a valid transaction after receiving the HUB message. Send a message to HUB confirming that it can receive assets from Zone1;
    \item The HUB sends a message to Zone2, sends the asset to Zone 2, and the asset transaction is completed.
\noindent Due to an attack or network error during the transfer process, it is possible for the message sent by Zone 2 to the HUB is lost, as shown in the figure\ref{fig:timeout} below, after waiting for a period of time, the HUB sends a message telling Zone1 that the current transaction is timeout and the transaction fails.
\end{enumerate}
        \begin{figure}[H]
        \includegraphics[width=1\textwidth]{./figures/IBC_timeout.png}
        \centering
        \caption{IBC sequence diagram, timeout}%\protect\footnotemark}
        \centering
        \label{fig:timeout}
        \end{figure}
        
\noindent Cosmos based on \textbf{Tendermint} consensus which is the combination of PBFT and PoS, it improves the processing power of Cosmos Network. The cross-chain transaction between Cosmos and other heterogeneous blockchains need to be carried out through Cosmos Bridge, Bridge-Zone will be responsible for the docking with the original chain, including the confirmation of the original chain transaction, create or destroy the corresponding tokens.\\
\noindent Overall, Cosmos is a blockchain development framework, allowing developers to focus on their own business without having to consider the underlying technology of the blockchain so the plug-in design can use Cosmos as needed.


\subsubsection{Polkadot}
\noindent Polkadot\cite{polkadot} is a cross-chain project supported by the Web3 Foundation. Creating a scalable, heterogeneous, multi-chain architecture designed to address blockchain interoperability, scalability, and security.\\
\noindent The overall architecture of Polkadot is shown in the figure below, consisting of a Relay Chain, Parachains and Bridges. 
\begin{itemize}
    \item \textbf{Relay chain} is the central system of the Polkadot network, which coordinates the consensus and transactions between the chains, records account information and transaction status; 
    \item \textbf{Parachains} can be built by developers to collect and process transactions, and transfer to Relay chain;
    \item \textbf{Bridges} connect other heterogeneous blockchains (such as Ethereum) with Polkadot network.
\end{itemize} 
\noindent There are four roles in Polkadot's network:
\begin{enumerate}
    \item Collators (collecting user transactions, validating and submitting to Validators),
    \item Validators (validating and broadcasting transaction data committed by Collators, verifying blocks and paying deposits)
    \item Nominators (investment deposits selected for trust Validators)
    \item Fishermen (supervising the evil behavior in the network). 
\end{enumerate}
\noindent Collators and Validators are the performers of the main transaction when doing cross-chain operations, while Nominators and Fishermen are the participants in maintaining system trust.\\


\begin{figure}[H]
    \includegraphics[width=1\textwidth]{./figures/Polkadot.jpg}
    \centering
    \caption{Polkadot network architecture \protect\footnotemark}
    \centering
\end{figure}
\footnotetext{Image courtesy of Polkadot white paper\cite{polkadot}}

\noindent Parachain can share trust of the entire Polkadot network, but it also gives a certain confirmation right to the Relay chain; when the user's transaction information on Parachain A is transmitted to Parachain B, the process is as follows:
\begin{enumerate}
    \item The initiated transaction is sent to the Collator on Parachain A;
    \item Collator validates the transaction and packs it into the block; 
    \item Collator submits the block and state transition proof to the Validator on Parachain A;
    \item The Validator verifies the block it receives that contains only valid transactions and pays a certain amount of deposit;
    \item  After enough Nominators have paid a deposit for the Validator, they broadcast the block to the Relay Chain;
    \item The transaction of data has been executed.
\end{enumerate}
\noindent Unlike Cosmos, Polkadot supports cross-chain interoperability between heterogeneous chains. It does not only support asset transactions, but also the data transfer. Its system complexity and implementation are more difficult than Cosmos.

\subsubsection{ICON}
\noindent ICON\cite{icon} is committed to building a cross-chain network that connects all types of blockchain systems, enabling DApps to be interconnected across all types of blockchains. The cross-chain transactions primarily handled through notary mechanism. Figure \ref{fig:concept} below shows the overall conceptual model of ICON. With ICON, blockchains are connected around the \textbf{Nexus}, which is a loopchain-based blockchain. The whole system based on loop fault tolerant mechanism which is an enhancement of BFT-based algorithm.  \\


 \begin{figure}[H]
        \includegraphics[width=1\textwidth]{./figures/iconconcept.png}
        \centering
        \caption{{Conceptual model of ICON}\protect\footnotemark}
        \centering
        \label{fig:concept}
        
        \end{figure}
\footnotetext{Image courtesy of ICON white paper\cite{icon}, Nexus is a Multi-Channel blockchain comprised of Light Client of respective blockchains. Each blockchain connected to Nexus via a portal.}
        \begin{figure}[H]
        \includegraphics[width=1\textwidth]{./figures/icontrans.png}
        \centering
        \caption{{Assets transaction process through Nexus}\protect\footnotemark}
        \centering
        \label{fig:icon}
        
        \end{figure}
\footnotetext{Image courtesy of ICON white paper\cite{icon}}
\noindent Although rated as one slow progress cross-chain project, ICON still has the ambition to not only connect blockchains together but also aiming at realizing communication between the traditional ledger system and the blockchain world.


\subsubsection{AION}
\noindent Among the Interoperability Alliance\protect\footnotemark
 \footnotetext{Wanchain, ICON and AION form an alliance to overcome the technical difficulties towards cross-chain technology. reference: https://medium.com/helloiconworld/blockchain-interoperability-alliance-icon-x-aion-x-wanchain-8aeaafb3ebdd}
, AION differs from Wanchain and ICON. While the other two projects still focus on the assets and value transactions cross-chain, AION also expanding the business logic and interoperability between chains. Figure\ref{fig:aion} represents a simple multi-tier network architecture based on AION.
        \begin{figure}[H]
        \includegraphics[width=1\textwidth]{./figures/aion.jpg}
        \centering
        \caption{{Multi-tier network structure of AION}\protect\footnotemark}
        \centering
        \label{fig:aion}
        
        \end{figure}
\footnotetext{Image courtesy of AION white paper\cite{aion}}
\noindent AION aims to establish a multi-tier cross-chain platform that supports interoperability of heterogeneous chains. By connecting networks and bridges to blockchain systems, the route of cross-chain transactions is a multi-stage process. \\
\noindent At each stage, the Validators verifies the transaction and agrees on whether the transaction is forwarded or rejected. Bridge validators will use a lightweight BFT-based algorithm to reach consensus. If a transaction is rejected at any time, any state changes due to cross-chain transactions will be revoked, at least in the connected network.\\
\noindent AION has different consensus mechanisms at different product stages, the latest AION 3.0 using mixed DPoS and PoI algorithm.\\
\noindent AION is an emerging project and still in the early stage of development.

