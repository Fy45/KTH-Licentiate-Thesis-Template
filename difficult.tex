\section{Difficulties}
\label{sec:diff}
\noindent The number one feature in the blockchain is immutability, and every record must be accurate to protect the value. Hence, in the cross-chain project, the key point is to ensure the accuracy of each transaction. There are many obstacles cross-chain need to encounter. To sum up, the following are several key points:\\


\begin{itemize}
    \item \textbf{How to ensure the atomicity of transactions.}\\
    The transactions should be atomic, that is, cross-chain transactions either occur or do not occur. Otherwise, the inconsistency and "out-of-sync" status of the two chains will become the most significant system vulnerabilities in cross-chain transactions, and the security of both systems will be threatened. The consistency is the fundamental guarantee for realizing cross-chain transactions, and it is also a keystone that must be solved in cross-chain transactions.
    \item \textbf{How to complete the confirmation of the transaction for other chains.} \\
    The confirmation of the transaction contains two levels of problems. One is to confirm that the transaction has occurred, wound up and written into the right block. Second is to make sure the system has approved the transaction with enough blocks. In this way, the probability of invalidation of the transaction due to system reconfiguration will be very low. The blockchain system itself is relatively isolated, lacking the mechanism to obtain external information actively. Therefore, it is not an easy task to confirm the transaction status of another chain. It can be said that it is one of the core difficulties of cross-chain transactions.
    \item \textbf{How to ensure that the total assets of the two chains remain unchanged.} \\
    In the scenario of asset exchange, the assets of the two chains are not substantially exchanged, so this type of situation does not change the total assets of each chain. However, in the scenario of asset transfer, the number of available assets in each chain will be altered. The total assets of both chains can remain unchanged only when the cross-chain transactions are accurately recorded, and atomically occurred. 
    \item \textbf{How to ensure the independent security of the two chains.} \\
    When two chains interoperate, they inevitably affect each other with dependency issues. It is a considerable problem that how to ensure blockchain system security during the process of cross-chain communications. If the security issue cannot be isolated, then one attacked chain will affect the entire cross-chain network.
    \item \textbf{How to realize multiple chains interoperability.} \\
    Take the history of the computer network as a reference. The independent blockchain network will eventually embark on the future of interconnection. How to link these existing and future blockchain networks to be unified into one whole network will be one of the most important issues of the future cross-chain network. 
\end{itemize}

\section{Summary}

\noindent This chapter introduced a thorough background study of the cross-chain research area. It briefly reviewed the history of cross-chain development and outlined the importance of the growing technology of cross-chain. Two main manifestations of cross-chain transactions were proposed to give a classification standard for Chapter \ref{chap:3} project study. Chapter \ref{chap:2} also described the critical difficulties that cross-chain projects are facing. Some of them will be discussed in the following chapters.