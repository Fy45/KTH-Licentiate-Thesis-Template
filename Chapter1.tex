\chapter{Introduction}
\label{chap:1}




\section{Motivation}

\noindent Originally, the ledger is the foundation of accounting. However, it is not difficult to find that there are many shortcomings in the long-term use of traditional ledgers. For example, low efficiency, high cost, opacity and easy to cause fraud and abuse issues.\\
\noindent With the development of information technology, these ledgers have gradually evolved into digital technology. Distributed ledger is a major leap after the digitization of ledger-based technology. A distributed ledger is a database that is shared, replicated, and synchronized between network members. It records transactions between network participants, such as the exchange of assets or data. From a technical point of view, the Distributed Ledger Technology (DLT) not only inherits the traditional bookkeeping philosophy but also has its unique innovations, which have some advantages that traditional ledgers cannot reach.\\
\noindent A single blockchain network is a relatively closed system that does not actively interact with the outside world. The assets of each chain are also an independent value system. If we can break through the chain between the chains and let the value circulate in the wider world, it will inevitably increase the value of the certificate assets from a wider scope, thus promoting the rapid development of the blockchain industry. Cross-chain technology is dedicated to building a bridge of trust between chains and chains, breaking the situation of one chain and one island, and realizing asset interoperability between chains and chains, in order to achieve a true win-win situation.

\section{Research Objectives/Research Questions, Research Methodology}


\noindent Based on the background above, there exist many distributed ledgers and for them to be fully distributed they need to communicate with each other and also centralized ledgers. Otherwise, the consistency between the ledgers of different entities across the chain would not be guaranteed. Nowadays, many different approaches have been suggested but so far no solution that has gained traction in the market.
So the main question in this project is to find what are the needs and solutions for a universal cross ledgers communication scheme.\\
\noindent From the development of computer networks, we seem to have seen the future of blockchain cross-chain networks. The current blockchain world is like the single-machine era of the 1960s. Chains are highly isomerized and difficult to interoperate. All data and services are confined to the island-like blockchain. In the future, all blockchain systems can be linked through a standardized cross-chain protocol, and many blockchain systems can work together to support more users and more services. The maturity and popularity of cross-chain technology may ignite the prosperity of the blockchain network. The difference is that the Internet is a network of freely flowing information, and the blockchain cross-chain network is a free-flowing network.\\
\noindent Cross-chain projects have become an important track in the blockchain field, and more and more projects have been added.

\section{Research Contributions}

\noindent My research has contributed to the universal demanding and requirements on cross-ledger communication towards blockchain areas. In particular, I have focused on the problems that the realization of cross-chain communication is facing.  

\section{Thesis Organization}

This thesis is organized into five chapters as follows:

$\bullet $ Chapter \ref{chap:1} concentrates on the research value and market meaning of this thesis, then briefly introduces the concept of cross-chain communication.

$\bullet $ Chapter \ref{chap:2} studies the background of the cross-chain project by classifying different manifestations of cross-chain communication as well as pointing out the problems cross-ledger communication facing.

$\bullet $ Chapter \ref{chap:3} focuses on the theoretical solutions that will address the difficulties, discusses the communication process of various cross-chain projects based on different group rules. 

$\bullet $ Chapter \ref{chap:4} ...

$\bullet $ Chapter \ref{chap:5} summarizes the findings during the research and suggests several ideas for related future work.

$\bullet $ Appendix \ref{app:A} contains a comparison table that evaluating 20 cross-chain projects from several valuable aspects.