\begin{abstract}


\noindent The phenomenon of isolated value in each blockchain system has become a distinct issue of blockchain field. To address this problem, the demand of cross-chain intercommunication came up. In a narrow sense, cross-chain refers to the process of asset interoperability between relatively independent blockchains. In this thesis, we mainly analyze the design principles, technical difficulties, and solutions of cross-chain intercommunication in a narrow sense. With the introduction of distributed ledger technology(DLT), we describe the interaction with other ledgers as the fundamental problem of current blockchain technology. \\ 

\noindent The implementation form of cross-chain is mainly manifested as asset swap and asset transfer. There are existing applications in the industry so far, and various cross-chain application scenarios can evolve from this. This paper will focus on these two implementations, illustrate their principles, located the realization difficulties, and put forward corresponding possible solutions. Then we elaborated 8 popular cross-chain projects underlying mechanism listed with three main categories. A detailed comparison according to their interoperability level, consensus algorithm and application scenarios of the overall overview of 20 cross-chain projects is presented as a table in the~\nameref{app:A}.\\

\noindent During the implementation process, we performed a simple atomic swap cross-chain framework based on Hash Time Lock Contract between Bitshares and Ethereum. Compared with the performance of one wallet application test case using Interledger protocol. These two applications are represented the two manifestations of cross-chain realization.\\

\noindent With limited projects to test out, our conclusion was reached after a discussion with relative merits of two approaches practically. Interledger protocol has a better solution from the aspects of the decentralization, scalability, and whether it supports with traditional ledgers. \\

\textbf{Keywords:} Distributed Ledger Technology, Cross-chain study, Atomic swaps, Interledger protocol, Relay, Sidechains
\\
\\
\\
\\
\clearpage
\end{abstract}
\endinput
