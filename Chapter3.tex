\chapter{Project Classification} 
\label{chap:3}

  

\section{Solutions}
\label{sec:sol}
\noindent Based on what we discussed about the difficulties in the previous sections, some solutions came up with multiple cross-chain projects. The following paragraphs will illustrate them thoroughly.
\subsection{Ensure the atomic of transactions}

\subsubsection{Atomic swaps}
\noindent An atomic cross-chain swap\cite{herlihy2018atomic} is the basic theoretical framework for multiple parties exchanging assets across multiple blockchains. Atomic operations in computer science ensure every exchange either success or failure, no third intermediate state.\\
\noindent For a more intuitive introduction to the atomic swap protocol, we assume an exchange scenario shown in Figure~\ref{fig:atomic}:

   \begin{figure}[htb!]
    \includegraphics[width=0.8\textwidth]{./figures/atomic_swaps.png}
    \centering
    \caption{Atomic swaps diagram}%\protect\footnotemark}
    \centering
    \label{fig:atomic}
    \end{figure}
\begin{enumerate}
  \item Assume Alice has 1 BTC on Bitcoin while Bob has 20 ETH on Ethereum. Alice wants to exchange Bob's 20 ETH for 1 BTC. Both Alice and Bob have wallet addresses on both Bitcoin and Ethereum.
    \item To start this transaction, Alice needs to randomly generate a key $K$, which is known only to Alice. Then she initiates a 1 BTC on-chain transaction (transaction \textcircled{1}) to Bob. The transaction can only be finished when Alice obtains the signature of Bob and provides the key $K$.
    \item Before broadcast transaction \textcircled{1}, Alice will first broadcast a retracement (transaction \textcircled{2}). If transaction \textcircled{1} does not receive the correct key or signature within 48 hours, the amount paid by that transaction will be returned to Alice. Transaction \textcircled{2} must be signed by Alice and Bob after broadcasted to take effect. At the same time, Alice will only broadcast the transaction \textcircled{1} to the network if transaction \textcircled{2} is successfully validated.
    \item Bob now sees the transaction \textcircled{2} sent by Alice. If Bob agrees, he will sign the transaction \textcircled{2}, of course, Alice will also complete the signature so that the retracement will take effect. Then Alice will broadcast the transaction \textcircled{1} to the whole network.
    \item Bob can only get the value $K$ after he pays Alice with 20 ETH. Hence Bob initiates transaction \textcircled{3} on Ethereum to pay Alice 20 ETH. These 20 ETH are only available if Alice enters the decrypted key $K$ and attaches Alice's signature. To prevent Alice from denying, Bob also issues a retracement transaction \textcircled{4} that requires Alice and Bob to sign contract together before networks broadcast transaction \textcircled{3}, when Alice does not provide the correct key or the signature within 24 hours. Then activate the retracement, 20 ETH will be returned to Bob.
    \item After Alice sees the transaction \textcircled{4}, both Alice and Bob need to attach their signature to this transaction to take effect. At this time, Bob will broadcast transaction \textcircled{3} to network.
    \item To get 20 ETH, Alice will sign transaction \textcircled{3} with the correct value $K$. For now, transaction \textcircled{3} succeeds, Alice obtains 20 ETH, and Bob obtains key $K$.
    \item After Bob gets the key $K$, he goes back to Bitcoin, enters the key $K $and his signature, and finally gets 1 BTC from Alice.
\end{enumerate}
\noindent From the diagram, it should be noticed that the atomic swap protocol does not transfer the assets of Bitcoin to Ethereum, but only the assets ownership of both chains. The total assets of Bitcoin and Ethereum have not changed, so it can only achieve asset exchange between chains and cannot achieve asset transfer.\\

\noindent This solution not only can be applied to the decentralized ledger system but also the centralized ledgers. As long as the two systems provide the functions of retracement, time lock, and key lock.

\subsubsection{Hash Time-Locked Contracts(HTLC)}
\noindent HTLC~\cite{russell2015lightning} is a very technical implementation of the atomic swap protocol. It guarantees the atomicity of the transaction through the hash lock and the time lock mechanism. In different systems, whether it is a blockchain system or a centralized ledger system, despite the ways of implementing the lock, the principle behind it is the same, that is, only certain hash conditions or time met, the transaction is allowed to take effect.

        \begin{figure}[H]
        \includegraphics[width=0.8\textwidth]{./figures/Hashlock.png}
        \centering
        \caption{Hash Time-lock Contract diagram}%\protect\footnotemark}
        \centering
        \label{fig:hash}
        \end{figure}
\noindent Using only hash time locks is not enough when you want to achieve cross-chain asset transfer. You also need to cooperate with other cross-chain technologies to ensure the authenticity of cross-chain transactions.


%\subsubsection{Hash Time-locked Agreements(HTLA)}
%\noindent HTLA is one HTLC generalization protocol came up by Interledger\cite{HTLA}, regardless whether the system support HTLC or not, whether it's a distributed or centralized ledger. HTLA can be used to implement cross-chain exchange between system or even support multi-hop cross-chain interchange between multiple systems, as shown in Figure \ref{fig:HTLA} below.
%   \begin{figure}[H]
%    \includegraphics[width=0.8\textwidth]{./figures/HTLA.png}
%    \centering
%    \caption{Interledger HTLA diagram}%\protect\footnotemark}
%    \centering
%    \label{fig:HTLA}
%    \end{figure}
%    
%\noindent Alice and Bob can trade among blockchains A, B, and C via HTLA, and each blockchain supports different cross-chain protocols. The connector here plays a role in connection and isolation. Linking blockchains that support different cross-chain protocols together, and again isolate them, so that blockchains do not interfere with each other.\\
%\noindent HTLA supports multiple cross-chain protocols based on HTLC; some of them are mentioned in Figure \ref{fig:HTLA}.


\subsection{Complete the transaction confirmation}
\noindent As we all know, blockchain systems are relatively independent and closed, there is no direct communication way for them to confirm every piece of records that happened. So no matter how it evolves, there will always be a "middle-man" between the two chains, taking on the role of information exchange between the two chains.\\

\noindent Here the "middle-man" represents any entity that could interact with two chains, it may be a one or a group, the centralized or distributed agency, a separate chain or even a functional module. The "middle-man" usually acts as a node for two blockchains at the same time, so that the requirement can only be one application software deployed on the same node to obtain the others' system data.\\

\noindent After the "middle-man" completes the data collection, how to confirm the transaction, which transaction to confirm, and who confirms become the key points of this problem. According to different schemes, this process can be summarized in three ways:

\begin{itemize}
    \item \textbf{Notary~\cite{buterin2016chain}}:\\
    In the notary scheme, a trusted one or group is used to declare to the chain A that an event has occurred on the chain B, or that the statement is correct. These groups can both automatically or requested to listen and respond to events. There are three different sub-schemes came up in the evolution of this model: 
    \begin{itemize}
        \item \textit{Centralized Notary schemes}\\
        The centralized notary mechanism is also called the single-signature notary mechanism, usually played by a single designated independent node or institution, which is the simplest model. It purpose a scheme that instead of letting Alice and Bob to perform trade directly, the reliability is increased when dealing indirect transactions with third-party institutions with credit endorsements (such as Alipay) both parties trusts. Since Alice and Bob exist in different ledger systems, the notary is technically required to be compatible with two or more systems at the same time.
        \begin{figure}[H]
        \includegraphics[width=0.6\textwidth]{./figures/cnotary.png}
        \centering
        \caption{Centralized Notary Scheme diagram}%\protect\footnotemark}
        \centering
        \label{fig:cno}
        \end{figure}
        To some extent, the use of centralized institutions has replaced technical credit guarantees, from professional credibility to traditional credit intermediaries. Although this kind of mode has fast transaction processing, strong compatibility, and simple technical architecture, the security of the central node has become a critical bottleneck for system stability.
        \item \textit{Multi-sig Notary schemes}\\
       The multi-signature notary mechanism is accomplished by multiple notaries that can sign a common agreement on their respective ledgers to complete the cross-chain transaction. Each node of the multi-signature notary group has its private key, and cross-chain transactions can only be confirmed when a certain number or proportions of notary signatures are reached.

        \begin{figure}[H]
        \includegraphics[width=0.675\textwidth]{./figures/mnotary.png}
        \centering
        \caption{Multi-sig Notary Scheme diagram}%\protect\footnotemark}
        \centering
        \label{fig:mno}
        \end{figure}
       This method is more secure than the single-signature mode, and a few notaries who are attacked or do evil will not affect the regular operation of the system. However, this approach requires both chains to have the ability to support multiple signatures.
       
        \item  \textit{Distributed signature Notary schemes} \\
        The main difference between a distributed signature and multi-signature is the signature generation. Distributed signature using \textit{Multi-Party Computation}(MPC), which will enhance the security as well as the implementation difficulty.
        \begin{figure}[H]
        \includegraphics[width=0.675\textwidth]{./figures/dnotary.png}
        \centering
        \caption{Distributed signature Notary Scheme diagram}%\protect\footnotemark}
        \centering
        \label{fig:dno}
        \end{figure}
        As Figure \ref{fig:dno} shows, distributed signature is based on cryptography, the key point is that for cross-chain transactions, the system generates one and only one key. No one in the notary group will have a complete key. The key is randomly sent to each notary node in the form of fragments.
        Meanwhile, the fragment is processed as cipher-texts, making it impossible to extract the complete key even all the participants put the pieces together. Thus, the security of the key is fully guaranteed.
    \end{itemize}
    
    \item \textbf{Relay\cite{buterin2016chain}}:\\
    Relay is one flexible and easy-to-expand cross-chain technology that does not rely on trusted third parties to help with transaction verification. Instead, it is self-verified by the receiving chain after receiving the data. Self-verification methods are depending on the system structure. For example, BTC-relay\cite{btc-relay} based on \textit{Simplified Payment Verification}(SPV), and Cosmos~\cite{cosmos} rely on verify the number of nodes' signature.\\
    
    Vitalik mentioned relay in his Chain Interoperability paper~\cite{buterin2016chain}, pointing out that chain A and chain B can use the other party's block data for information synchronization and cross-chain function calls. Currently, information synchronization can be done, but there is no mature technical solution for cross-chain function calls. Two chains cannot verify the validity of each others block at the same time. Otherwise, they will fall into an infinite loop of nesting. 
%    If chain A owns the block data of chain B, then chain A needs to be confirmed in the case of chain B transaction confirmation, and chain B needs to wait for chain A's transaction confirmation because it also has block A's block data, it goes on as a loop.
    
    
    
    \item \textbf{Sidechains}:\\
     The concept of a sidechain as defined in white paper~\cite{back2014enabling} is: \textit{sidechain is a blockchain that validates data from other blockchains}. However, this explanation was considered to be too broad and not rigorous by Vitalik Buterin in Chain Interoperability~\cite{buterin2016chain}. "Sidechain" is more frequently  refered to what Blockstream calls a "pegged sidechain". Pegged chain has the function of anchor the child assets to the parent chain. In this way, this relationship is based on assets, not the blockchain itself. Pegged sidechain is a strong coupling cross-chain structure using directly embeds part of the data of the original chain into its own block or storage space. In the case of cross-chain transactions,  verification can be completed directly through the original blockchain data stored in the system. This method is generally considered bidirectionally at the beginning of the system design.\\
     
     Compared to notary and relay, the sidechain is more direct. The state of one chain will be directly reflected in the data of the other chain. When one chain is attacked, the other chain may also be affected. This model is more suitable for the design of same systems, which allows the two sides to become a whole part without losing the relative independence of the ledgers.

\end{itemize}
\subsection{Realize multiple chains interoperability}
%\noindent The independent computer network connects with each other into a local area network, the local area network develops into a metropolitan area network, the metropolitan area network evolves into the Internet, and the Internet connects the people of the world like never before.\\

\noindent Unlike the evolution of computer network, it is still in the "single-machine" era for the emerging technology of blockchain, and the interactions demand between chains will become increasingly strong with the application of blockchain.\\
\noindent To realize interoperability among multiple chains, there are two potential aspects of difficulties that need to be overcome: 
\begin{itemize}
    \item How to achieve interoperability among blockchains system that has already developed.
    \item How to prepare and setup the way for the interconnections among the new blockchains in the future.
\end{itemize}

\subsubsection{Active compatibility}
\noindent This solution is mainly aimed at the existing blockchain system. First, there are existing different blockchain application systems in the upper layer, and then the underlying cross-chain mechanism is developed.\\
\noindent Usually these systems are heterogeneous and need to be docked one by one, but there is also a different solution for a pair of connections.
\begin{enumerate}
    \item Direct interconnection between the two chains
        \begin{figure}[H]
        \includegraphics[width=1\textwidth]{./figures/direct.png}
        \centering
        \caption{Direct interconnection network architecture diagram}
        \centering
        \label{fig:direct}
        
        \end{figure}
   This method is the most time-consuming and laborious without the support of the unified underlying protocol. It is necessary to establish six paths between the four chains to realize the interconnection between them as shown in Figure~\ref{fig:direct}. Also, each path needs to be customized. Although this method is not scalable, it can guarantee better security and independence. Once an attack occurs, it is difficult to affect the entire network.
    \item Third-party cross-chain platform
      \begin{figure}[H]
        \includegraphics[width=1\textwidth]{./figures/platform.png}
        \centering
        \caption{Cross-chain platform network architecture diagram}
        \centering
        \label{fig:platform}
        
        \end{figure}
  Figure~\ref{fig:platform} shows that once the cross-chain platform established, blockchains can indirectly interconnect with each other. Thus, only four paths are required to build a cross-chain network. However, in this method, the cross-chain platform will become the key point and performance bottleneck of the entire cross-chain network. Once the cross-chain platform is attacked, the entire cross-chain network will be paralyzed.
\end{enumerate}
\subsubsection{Passive compatibility}
\noindent Passive compatibility is mainly aimed at the blockchain system that has not been developed. It first builds the underlying cross-chain platform, allowing other blockchain systems to be easily, conveniently, and securely accessed.\\

\noindent Cross-chain platforms will prioritize the development of systems and protocol standards that apply to interoperability between the various chains. The subsequent development of standards-compliant development on existing platforms allows for the creation of blockchains that naturally have cross-chain functionality within the system. However, the cross-chain mentioned here refers to the chain that conforms to the protocol standard, which can be easily connected. If it is to interoperate with other chains outside the system, it is necessary to develop a separate middleware to communicate.\\

\noindent Besides, different cross-chain platforms can support different types of blockchains, such as Cosmos supporting isomorphic chains and Polkadot supporting heterogeneous chains, both of which are highly scalable.


\section{Project study}
\label{sec:p}

\subsection{Lightning Network}
\noindent In general, we can not say lighting network realize the cross-chain function, though it provided a classic application towards atomic swaps and HTLC. The design idea of lighting network is very simple, it put a large number of high-frequency small-value transactions off-chain to expand the transaction processing capability of the blockchain.\\
\noindent Lightning network\cite{poon2016bitcoin} is a fast and scalable Bitcoin transaction project, it has two main technical points:
\begin{enumerate}[A.]
    \item \texttt{Recoverable Sequence Maturity Contract} \\
    RSMC is similar to a reserve mechanism in which both parties trade in an off-chain trading pool. A certain amount of assets is used as the mutual funds for the transactions between the two parties, and the share of the assets is recorded off-chain. This trading pool is a “micro-payment channel”. When a transaction occurs between two parties, the proportion of the common assets in the trading pool will change. The new proportional data needs to be signed and confirmed by both parties, and the old proportional share version is invalid. The entire process is done off the chain, so it does not occupy the resources of the main chain. The final proportion of assets will be confirmed and record to the main-chain after one of the transaction party requires a withdraw.\\
    It could happen anytime as long as both parties signed for this. To ensure the security, if someone submitted the old share of assets to make profits, others could protect themselves by proving this balance sheet is not the latest one. Then the asset of the counterfeit party will be confiscated to the challenger.
    \item \texttt{Hash Time-locked Contract}\\
     Lightning Network uses HTLC to guarantee the atomicity of transaction as shown in Figure \ref{fig:hash}. This diagram simply illustrates how HTLC works to provide limited time transfer function. The basic process is: Bob and Alice can reach an agreement that specifies Alice to lock a certain amount of assets and provide a hashed value $H$. Before the arrival of time $T$, if Bob can learn an appropriate $s$ (secret) where its hash value matches with $H$ and sent to Alice, Bob can get the corresponding amount of assets value. Conversely, the asset will unlock and return to Alice.
\end{enumerate}
\noindent When there are “micro-payment channels” between multiple users, these channels are connected to each other to form a “channel network”, which is the lightning network. The mutual transfer of the two parties does not require a direct payment channel to connect to each other, but also through the intermediary to achieve mutual transfer. \\ \\
\noindent \begin{large}
\textbf{Disadvantages}:
\end{large}
\begin{itemize}
    \item Users cannot pay off-line. 
    \item More suitable for small transactions. Large transfer amount needs to open multiple channels.
    \item Easy to face the dis-matched situation, if there's no response from one party, the other one may need hours to close the channel and substitute with another route.
\end{itemize}


\subsection{Notary Scheme}
\label{sec:notary}
\subsubsection{Ripple Interledger protocol}
\noindent Earlier we focused on cross-chain, this project is more focused on cross-ledger, which means that the agreement supports not only decentralized blockchains but also supports various centralized ledgers, which is broader support for cross-chain applications. Ripple ILP\cite{thomas2015protocol} uses the HTLA and Notary scheme to implement this technology.\\
\noindent Ripple is the first project to propose the use of blockchain technology to achieve cross-ledger exchange of assets, with a focus on resolving cross-border remittances, enabling faster and more economical international remittances via the Ripple network.\\
\noindent Interledger Protocol(ILP) is compatible with any online ledger systems. Specifically, the ILP will establish a two-way pegged relationship between the trader's account and a Ripple local account, enabling simultaneous changes between the two to ensure transparency in the transaction process. At the same time, for two ledger systems that do not have a direct payment channel, multi-hop indirect cross-ledger transactions can be realized through ILP. \\
\noindent The main idea of ILP is to secure cross-ledger transactions by setting up \textit{escrow account} on Ripple. So the process will need the preparation of escrow account of several parties in the transaction. As an example, in Figure \ref{fig: ILP}, Alice, Bob, and one selected market maker should have their Ripple escrow accounts set up on two bank systems before the transaction. \\
\begin{itemize}
    \item Alice first selects a market maker with the most suitable exchange rate, and fill in the remittance information, receipt address and timeout period on the Ripple application.
    \item The Interledger Module will pack this information and sent to the Ripple Account 1, Ripple Account 1 records the changed amount of currency in the escrow account 1 and sends the transfer certificate to the \textbf{Validator}
    \item For Bob, Company B fills in the Ripple application with information such as the remittance address and timeout period and broadcasts it on the Ripple network. At this time, the liquidity provider selected by A will transfer a certain amount of assets in B's currency from Ripple Acc.3 to Ripple Acc.2 in advance, then send the transfer certificate to the \textbf{Validator}
    \item Validator checks the two transfer certificates; after the verification is passed, the ILP ledger will be liquidated simultaneously according to the Hashed Time Lock Agreement.
    \item The final step is when liquidation completed, Ripple will synchronize all account changes through an interledger module, thus realizing the cross-ledger transactions.
\end{itemize}
        \begin{figure}[H]
        \includegraphics[width=1\textwidth]{./figures/ILP.png}
        \centering
        \caption{ILP example process}%\protect\footnotemark}
        \centering
        \label{fig:ILP}
        \end{figure}
        

\subsubsection{Wanchain}
\noindent Wanchain\cite{wanchain.org} is a cross-chain platform project initiated in 2016. It is a heterogeneous cross-chain framework that implements cross-chaining based primarily on distributed notary scheme. This model mainly uses cryptography "Secure Multi-Party Computation" and "Threshold Key Sharing Scheme" to achieve the Authenticator distributed signature.\\

\noindent Wanchain provides the infrastructure for asset cross-chain transfer channels for different blockchain networks, realizing the transfer of assets between Wanchain and original chain. Wanchain3.0 now launches bridges from Bitcoin to the Ethereum network. The transaction reliability verification is completed by multiple Storeman of Wanchain. The following figures~\ref{fig:wan1} and~\ref{fig:wan2} are shown the transfer process between Wanchain and Ethereum.

        \begin{figure}[H]
        \includegraphics[width=0.96\textwidth]{./figures/ethtowan.png}
        \centering
        \caption{{Data transfer process from Ethereum to Wanchain}\protect\footnotemark}
        \centering
        \label{fig:wan1}
        
        \end{figure}
\footnotetext{Image courtesy of Wanchain white paper\cite{wanchain.org}}
        \begin{figure}[H]
        \includegraphics[width=1\textwidth]{./figures/wantoeth.png}
        \centering
        \caption{{Data transfer process from Wanchain to Ethereum}\protect\footnotemark}
        \centering
        \label{fig:wan2}
        
        \end{figure}
\footnotetext{Image courtesy of Wanchain white paper\cite{wanchain.org}}

\begin{enumerate}
    \item The token of the user in the original chain will be sent to Wanchain in the locked account, and Hash Time Lock locks the transaction;
    \item After the \textbf{Voucher} verified  the transaction on the original chain, the \textbf{Storeman} will initiate a cross-chain contract transaction on the Wanchain, and transfer the mapping token in Wanchain to the user's cross-chain account on Wanchain, and locked;
    \item After the user's wallet detects the transaction locked by the cross-chain contract, release the Secret to the cross-chain contract;
    \item Storeman obtains the control of the original chain token through the secret number, thus achieving confirmation of the original chain transaction.
    \item If the user does not release the secret number within the scope of the hash time lock, the hash time lock expires
    \item The transaction of the post-cross-contract is automatically invalidated, and the user regains control of the original token.
\end{enumerate}

\noindent In Wanchain, when Storeman locked an account, the private key of the locked account is scattered into multiple pieces and send to Storeman, and it requires more than a certain percentage of Storeman complete the signature before the final confirmation. To avoid conspiracy, Storeman has to pay a certain amount of token to participate in the verification in case of doing evil. To ensure atomicity, Wanchain uses a hash time lock to lock cross-chain transactions, ensuring that no user or Storeman will complete a one-sided transaction.\\

\noindent Since the Wanchain mechanism does not change the original chain. It is necessary to adapt the development according to the characteristics of the original chain, which is also the difficulty of heterogeneous cross-chain solution. The transaction speed is affected by the confirmation speed of the original chain.\\\\

\subsection{Relay Scheme}
\subsubsection{BTC-Relay}
\noindent In 2016, the BTC-Relay released by the Consensys team~\cite{btc-relay} is the most classic relay cross-chain solution, enabling cross-chain transactions between Ethereum and Bitcoin, as well as realizing Ethereum's Decentralized applications to support BTC payments. Since Bitcoin scripts are non-Turing complete and difficult to support complex applications, BTC-Relay only implements one-way cross-chain from Bitcoin to Ethereum.
        \begin{figure}[H]
        \includegraphics[width=1\textwidth]{./figures/btc.png}
        \centering
        \caption{BTC-Relay cross-chain process diagram}%\protect\footnotemark}
        \centering
        \label{fig:btc}
        \end{figure}
\noindent Like Figure \ref{fig:btc} shows, BTC-relay itself is a smart contract for Ethereum. The function of the contract is to verify certain transactions on Bitcoin and provide verification information to other DApp users on the Ethereum. \\

\noindent Relay is a group of users who obtain block header data from Bitcoin and has the account address of the Ethereum network. The Relay that submits the block header data to the BTC-Relay contract as soon as possible can get the ETF transaction fee reward. After obtaining the block header data, the BTC-Relay smart contract can verify a transaction according to the principle of SPV proof. When a transaction in the Bitcoin network does occur, it can trigger the specific transaction or smart contract execution of the Ethereum network.

\subsubsection{Cosmos}
\noindent Cosmos~\cite{cosmos} is a cross-chain platform project initiated by the Tendermint team in 2017. It supports the modular establishment of Cosmos isomorphism chain and also supports the external heterogeneous chain through Bridge. Its most important feature is that all the chains in the Cosmos system are isomorphic and can more easily support the flow of assets across the chain. All zones share a set of network protocols, consensus mechanisms, and data storage methods, assembling the new one through the API interface.\\

\noindent The overall architecture of Cosmos network are shown in Figure\ref{fig:cosmos}:

        \begin{figure}[H]
        \includegraphics[width=1\textwidth, height=2.3in]{./figures/cosmos.png}
        \centering
        \caption{Cosmos network architecture\protect\footnotemark[1]}
        \centering
        \label{fig:cosmos}
        \end{figure}
  \footnotetext[1]{Figure taken from: \url{https://golden.com/wiki/Cosmos_network}}
        
\noindent There are many Zones connected to the Hub (both are chains). Cosmos Hub maintains a multi-asset distributed ledger and masters the asset status of all the Zones that connected to it. Each Zone will synchronize the state of the Hub, but the communication among Zones can only be done indirectly through the Hub. Each cross-chain asset transfer requires a standard successful confirmation from Zones and Hub. \\

\noindent The information is transmitted between zones through packets based on the IBC (Inter-Blockchain Communication) protocol. The blocks in a space pack the data to be delivered into standard IBC data packets and finally complete the transmission through the network layer with UDP or TCP protocol.
        \begin{figure}[H]
        \includegraphics[width=1\textwidth]{./figures/IBC.png}
        \centering
        \caption{IBC sequence diagram}%\protect\footnotemark}
        \centering
        \label{fig:IBC}
        \end{figure}
\begin{enumerate}
    \item Zone 1 initiates an IBCBlockCommitTx transaction, passing the new block header information (including all Validator's public key) to the Hub;
    \item Zone 1 initiates an assets transaction and verifies it validity;
    \item Zone 1 sends the valid transaction into the message queue for Hub;
    \item Zone 1 listens to a new message in the queue, which generates Merkle proof, and sent to Hub as IBCPacketTx's Payload. (In each space there is an independent third-party relay program that produces Merkle proof from the original chain and assembles it into a packet, initiates the transaction, passing it to the receiving chain);
    \item Hub verifies that Merkle proof is valid, send a message to Zone2 (The process of sending a message to Zone 2 by Hub is the same as previous steps);
    \item Zone 2 verifies that Zone 1 has a valid transaction after receiving the Hub message. Send a message to Hub confirming that it can acquire assets from Zone 1;
    \item Hub complete one asset transaction by messaging Zone 2 and transfer the assets.
\noindent Due to an attack or network error during the transfer process, it is possible for the message sent by Zone 2 to the Hub is lost, as shown in the figure\ref{fig:timeout} below, after waiting for a period of time, the Hub sends a message telling Zone 1 that the current transaction is timeout and the transaction fails.
\end{enumerate}
        \begin{figure}[H]
        \includegraphics[width=0.8587\textwidth]{./figures/IBC_timeout.png}
        \centering
        \caption{IBC sequence diagram, timeout}%\protect\footnotemark}
        \centering
        \label{fig:timeout}
        \end{figure}
        
\noindent Cosmos based on \textbf{Tendermint} consensus~\cite{buchman2016tendermint}, which is the combination of PBFT and PoS, it improves the processing power of Cosmos Network. The cross-chain transaction between Cosmos and other heterogeneous blockchains need to be carried out through Cosmos Bridge, Bridge-Zone will be responsible for the docking with the original chain, including the confirmation of the original chain transaction, create or destroy the corresponding tokens.\\

\noindent Overall, Cosmos is a blockchain development framework, allowing developers to focus on their own business without having to consider the underlying technology of the blockchain so the plug-in design can use Cosmos as needed.

%
%\subsubsection{Polkadot}
%\noindent Polkadot\cite{polkadot} is a cross-chain project supported by the Web3 Foundation. Creating a scalable, heterogeneous, multi-chain architecture designed to address blockchain interoperability, scalability, and security.\\
%\noindent The overall architecture of Polkadot is shown in the figure below, consisting of a Relay Chain, Parachains and Bridges. \\
%
%\begin{table}[H]
%\begin{tabular}{l|c}
%
%Role& Description \\
%\hline
%\multirow{3}{1in}{\textbf{Relay chain}}& The central system of the Polkadot network, which \\& coordinates the consensus and transactions between the chains, \\& records account information and transaction status; \\
%\hline
%\multirow{2}{1in}{\textbf{Parachains}}& Built by developers to collect and process transactions,\\&and transfer to Relay chain; \\
%\hline
%\multirow{2}{1in}{\textbf{Bridges}} & Connects other heterogeneous blockchains (such as Ethereum) \\& with Polkadot network.\\
%\hline
%\textbf{Collators} & Collects user transactions and submitting to Validators\\
%\hline
%\multirow{2}{1in}{\textbf{Validators}} &Validates and broadcasts transaction data committed\\ & by Collators, verifys blocks and paying deposits\\
%\hline
%\textbf{Nominators} &Investment deposits selected for trust Validators \\
%\hline
%\textbf{Fishermen} &Supervises the evil behavior in the network\\
%\hline
%\end{tabular}
%\caption{Main roles in Polkadot}
%\end{table}
%
%\noindent Collators and Validators are the performers of the main transaction when doing cross-chain operations, while Nominators and Fishermen are the participants in maintaining system trust.\\
%
%
%\begin{figure}[H]
%    \includegraphics[width=1\textwidth]{./figures/Polkadot.jpg}
%    \centering
%    \caption{Polkadot network architecture \protect\footnotemark}
%    \centering
%\end{figure}
%\footnotetext{Image courtesy of Polkadot white paper\cite{polkadot}}
%
%\noindent Parachain can share trust of the entire Polkadot network, but it also gives a certain confirmation right to the Relay chain; when the user's transaction information on Parachain A is transmitted to Parachain B, the process is as follows:
%\begin{enumerate}
%    \item The initiated transaction is sent to the Collator on Parachain A;
%    \item Collator validates the transaction and packs it into the block; 
%    \item Collator submits the block and state transition proof to the Validator on Parachain A;
%    \item The Validator verifies the block it receives that contains only valid transactions and pays a certain amount of deposit;
%    \item  After enough Nominators have paid a deposit for the Validator, and they broadcast the block to the Relay Chain;
%    \item The transaction of data has been executed.
%\end{enumerate}
%\noindent Unlike Cosmos, Polkadot supports cross-chain interoperability between heterogeneous chains. It does support not only asset transactions but also data transfer. Its system complexity and implementation are more difficult than Cosmos.

%
%\subsubsection{Quant Overledger}
%\noindent Developed by Quant Network, Overledger is the one Blockchain Operating System that facilitates the development of multi-chain smart contracts\cite{verdian2018quant}. In order not to limit the inter-communication between 2 blockchains at the same time, Overledger enables reading the transactional, contract and script information and map them into one ``over layer''. So in most cases where the transaction requires multiple hops during the route to the destination, Overledger will create a common interface among ledgers to solve this issue.\\
%\noindent Different from other projects who devote their cross-chain solution into the transactional layer,






\subsection{Sidechain Scheme}
\label{sec:side}


\subsubsection{Plasma}
\noindent As the second-layer expansion framework of Ethereum, Plasma has been proposed by \textsc{Joseph Poon} (founder of Lightning Network) and \textsc{Vitalik Buterin} (founder of Ethereum) in 2017 \cite{poon2017plasma}. The first thing to be clear is that Plasma is essentially a set of frameworks instead of a separate project. It provides an off-chain solution for a variety of different project real projects. \\
\noindent Layer 2 expansion of blockchains often known as off-chain expansion, similar to lightning network, this kind of expansion scheme does not need to modify the underlying protocol of the blockchain, but by transferring a large amount of frequent calculation work ``off-chain'', and submitting the calculation results to the ``main chain'' guarantee its finality as we discussed previously. Plasma works like blockchains in blockchains where anyone can create different Plasma on top of the underlying blockchain to support different business needs. \\
\noindent As an example of sidechain, Plasma derived from the general concept of symmetric 2-way pegged scheme to realize the transfer solution. The overall process of asset exchange is shown in figure \ref{fig:2way}. 
        \begin{figure}[H]
        \includegraphics[width=1\textwidth]{./figures/2way.png}
        \centering
        \caption{{2-way pegged sidechain diagram}\protect\footnotemark}
        \centering
        \label{fig:2way}
        
        \end{figure}
\footnotetext{SPV to verify that the transaction exists (recognized by other nodes on the blockchain)}
\begin{enumerate}
    \item When chain A wants to transfer the asset to chain B, it first needs to initiate a transfer transaction Tx1 (chain A's locking addr1 $+$ chain B's receiving address addr2), and the asset M1 is locked on the addr1.
    \item After Tx1 transaction is submitted, it is necessary to wait for a \textit{confirmation period} so that there are enough blocks and calculations to ensure that the cross-chain transaction Tx1 is confirmed, reducing the impact of refactoring on cross-chain transactions.
    \item After the confirmation period, the \textbf{SPV} certificate containing Tx1 will be sent to chain B. B knows that chain A has indeed initiated and locked asset M1, so it generates a corresponding amount of M2 on chain B according to a certain ratio. The value of M1 is transferred to M2 means the assets on chain A are transferred to chain B.
    \item After M2 is generated on B, it is necessary to wait for the \textit{competition period} before unlocking M2 to avoid double-spend attack in chain A reconstruction.
    \item After unlocking, M2 can freely circulation on chain B.
    \item The process of a transaction from chain B to chain A is similar to previous steps.
\end{enumerate}
\noindent Plasma supports multi-level sidechains and uses MapReduce mode to perform parallel computing, which greatly improves sidechain performance. The block header and hash data of the side chain will be sent to the main chain, and \textit{Proof of Fraud} can be used to ensure the correctness of the sidechain transaction.



\subsubsection{Elastos}
\noindent In order to ease the pressure on the main chain as well as provide better user experience for DApps, Elastos\cite{Elastos} adopted the main chain + sidechain architecture,  main chain only responsible for the circulation of the ELA while the DApps run on the sidechain. \\
\noindent In this scenario, the transfer of the assets between the main chain to sidechain is in a one-to-many relationship. It is feasible that the side chain only saves all the block header information of the main chain. If the main chain needs to save the block header information of all the side chains, it will lead to poor scalability. So Elastos uses asymmetric 2-way peg scheme based on SPV to realize the cross-chain function.\\
\noindent Assets from the main chain to the sidechain, Elastos using SPV proof to prove the transactions, while it secures the transfer using multi-signature notary scheme when transfer from sidechain to the main chain. We can regard this process as a combination of Plasma and Liquid.

\subsubsection{OneLedger}
\noindent As one cross-chain consensus protocol, OneLedger\cite{Oneledger} uses sharding and improved practical Byzantine fault-tolerant consensus. By creating sidechain, it can easily realize cross-chain interactions between individuals or business in OneLedger. OneLedger defines a three-layer consensus protocol to integrate different blockchain applications more efficiently.\\
\noindent Different from what we have discussed before, OneLedger as a sidechain, it realizes the synchronization of assets and values between the main chain and sidechain by applying multi-sig federation and drive-chain.\\
\noindent A drivechain\cite{lerner2016drivechains} gives custody of the locked coins to the miners, allowing them to (algorithmically) vote on when to unlock coins and where to send them. As Figure \ref{fig:drive} shows:
        \begin{figure}[H]
        \includegraphics[width=1\textwidth]{./figures/drive.png}
        \centering
        \caption{Drivechain  working diagram}%\protect\footnotemark}
        \centering
        \label{fig:drive}
        \end{figure}
\noindent According to OneLedger's white paper, the core of OneLedger is a set of consensus protocols that enable OneLedger to effectively integrate different blockchain products. As far as I understand that the protocol mentioned here is not a specific consensus protocol algorithm in the traditional sense, but a series of concepts and application scenarios. Among all 3 layers, I specifically studied the \textit{Public Chain Consensus} which apply on the atomic transfer between blockchains through OneLedger Network on the base layer.
        \begin{figure}[H]
        \includegraphics[width=1\textwidth]{./figures/oneledger.png}
        \centering
        \caption{{OneLedger sidechain architecture}\protect\footnotemark}
        \centering
        \label{fig:oneledger}
        
        \end{figure}
\footnotetext{Image courtesy of OneLedger white paper\cite{Oneledger}}
\noindent There's 2 steps in sidechain consensus algorithm:
\begin{itemize}
    \item \textbf{Round base pre-consensus}: Use to obtain a consensus proposal with more than 2/3 participants' votes.
    \item \textbf{Commit}: When the purposed block has reached a pre-consensus stage, it needs to drive to the public chain when necessary, and accepts the verification process. Once the proposal is accepted by both public chains, the new block will be officially `committed' to the OneLedger network, and once more than 2/3 of the participants complete the commits, the block is finalized. 
\end{itemize}





% \subsection{Interoperability Alliance \protect\footnotemark}
% \footnotetext{Wanchain, ICON and AION form an alliance to overcome the technical difficulties towards cross-chain technology. reference: https://medium.com/helloiconworld/blockchain-interoperability-alliance-icon-x-aion-x-wanchain-8aeaafb3ebdd}






\section{Summary}
\noindent This Chapter have presented a thorough case study of 14 representative cross-chain projects and classified them into several working schemes. Through the official white paper and other documents, the preceding sections explained the working process and theory of them using diagrams to help the understanding. Including some of my personal comments and thoughts towards them. The total work of comparison will be shown as a table in the Appendix%\ref{app:A}.
