\begin{abstract}

\begin{otherlanguage}{swedish}

\noindent Fenomenet isolerat v\"{a}rde i varje blockchain-system har blivit en distinkt  fr\aa ga om blockchainf\"{a}lt. F\"{o}r att hantera detta problem kom kravet p\aa \space interkommunikation mellan kedjor upp. I en smal mening h\"{a}nvisar tv\"{a}rkedjan till processen f\"{o}r interoperabilitet mellan tillg\aa ngar mellan relativt oberoende blockchains. I denna avhandling analyserar vi huvudsakligen designprinciper, tekniska sv\aa righeter och l\"{o}sningar f\"{o}r interkommunikation mellan kedjor i en smal bem\"{a}rkelse. Med introduktionen av distribuerad huvudboksteknologi (DLT) beskriver vi interaktionen med andra bokar som det grundl\"{a}ggande problemet med den nuvarande blockchain-tekniken.\\

\noindent Tv\"{a}rkedjans implementeringsform manifesteras huvudsakligen som tillg\aa ngsbyte och \"{o}verf\"{o}ring av tillg\aa ngar. Hittills finns det m\aa nga befintliga applikations scenarier och Pro projekt som antagits fr\aa n dessa manifestation. Detta dokument kommer att fokusera p\aa \space dessa tv\aa \space implementeringar, illustrera deras principer, lokalisera f\"{o}rst\aa elsessv\aa righeterna och l\"{a}gga fram motsvarande m\"{o}jliga l\"{o}sningar. Sedan utarbetade vi \aa tta popul\"{a}ra tv\"{a}rkedjeprojekt underliggande mekanism listade med tre huvudkategorier. En detaljerad j\"{a}mf\"{o}relse beroende p\aa \space deras driftskompatibilitetsniv\aa, konsensusalgoritm och till\"{a}mpningsscenarier av den \"{o}vergripande \"{o}versikten \"{o}ver 20 tv\"{a}rkedjeprojekt presenteras som en tabell i~\nameref{app:A}.\\

\noindent Under genomf\"{o}randeprocessen genomf\"{o}rde vi en enkel atomisk swap Cross-Chain ram baserad p\aa \space hash Time lock kontrakt mellan Bitshares och Ethereum, sedan j\"{a}mf\"{o}ra prestanda med en pl\aa nbok program som finns med Ripple med Interledger Protocol. Dessa tv\aa \space applikationer f\"{o}rest\"{a}lls de tv\aa \space olika anv\"{a}ndningsfallen av Cross-Chain genomf\"{o}rande.\\

\noindent Med begr\"{a}nsade projekt att testa, n\aa ddes v\aa r slutsats efter en diskussion med relativa f\"{o}rdelar med tv\aa \space metoder praktiskt taget. Interledger-protokollet har en b\"{a}ttre l\"{o}sning med avseende p\aa \space decentralisering, skalbarhet och huruvida det st\"{o}der traditionella bokar.
\\
\end{otherlanguage}

\textbf{Keywords:} Distribuerad Huvudboksteknik, Tv\"{a}rkedjestudie, Atombyte, Interledger-protokoll, Rel\"{a}, Sidokedjor
\\
\\
\\
\\
\clearpage
\end{abstract}
\endinput
